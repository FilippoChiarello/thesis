\documentclass[]{book}
\usepackage{lmodern}
\usepackage{amssymb,amsmath}
\usepackage{ifxetex,ifluatex}
\usepackage{fixltx2e} % provides \textsubscript
\ifnum 0\ifxetex 1\fi\ifluatex 1\fi=0 % if pdftex
  \usepackage[T1]{fontenc}
  \usepackage[utf8]{inputenc}
\else % if luatex or xelatex
  \ifxetex
    \usepackage{mathspec}
  \else
    \usepackage{fontspec}
  \fi
  \defaultfontfeatures{Ligatures=TeX,Scale=MatchLowercase}
\fi
% use upquote if available, for straight quotes in verbatim environments
\IfFileExists{upquote.sty}{\usepackage{upquote}}{}
% use microtype if available
\IfFileExists{microtype.sty}{%
\usepackage{microtype}
\UseMicrotypeSet[protrusion]{basicmath} % disable protrusion for tt fonts
}{}
\usepackage[margin=1in]{geometry}
\usepackage{hyperref}
\hypersetup{unicode=true,
            pdftitle={Text Mining Techniques for Knowledge Extraction from Technical Documents},
            pdfauthor={Filippo Chiarello},
            pdfborder={0 0 0},
            breaklinks=true}
\urlstyle{same}  % don't use monospace font for urls
\usepackage{natbib}
\bibliographystyle{apalike}
\usepackage{longtable,booktabs}
\usepackage{graphicx,grffile}
\makeatletter
\def\maxwidth{\ifdim\Gin@nat@width>\linewidth\linewidth\else\Gin@nat@width\fi}
\def\maxheight{\ifdim\Gin@nat@height>\textheight\textheight\else\Gin@nat@height\fi}
\makeatother
% Scale images if necessary, so that they will not overflow the page
% margins by default, and it is still possible to overwrite the defaults
% using explicit options in \includegraphics[width, height, ...]{}
\setkeys{Gin}{width=\maxwidth,height=\maxheight,keepaspectratio}
\IfFileExists{parskip.sty}{%
\usepackage{parskip}
}{% else
\setlength{\parindent}{0pt}
\setlength{\parskip}{6pt plus 2pt minus 1pt}
}
\setlength{\emergencystretch}{3em}  % prevent overfull lines
\providecommand{\tightlist}{%
  \setlength{\itemsep}{0pt}\setlength{\parskip}{0pt}}
\setcounter{secnumdepth}{5}
% Redefines (sub)paragraphs to behave more like sections
\ifx\paragraph\undefined\else
\let\oldparagraph\paragraph
\renewcommand{\paragraph}[1]{\oldparagraph{#1}\mbox{}}
\fi
\ifx\subparagraph\undefined\else
\let\oldsubparagraph\subparagraph
\renewcommand{\subparagraph}[1]{\oldsubparagraph{#1}\mbox{}}
\fi

%%% Use protect on footnotes to avoid problems with footnotes in titles
\let\rmarkdownfootnote\footnote%
\def\footnote{\protect\rmarkdownfootnote}

%%% Change title format to be more compact
\usepackage{titling}

% Create subtitle command for use in maketitle
\newcommand{\subtitle}[1]{
  \posttitle{
    \begin{center}\large#1\end{center}
    }
}

\setlength{\droptitle}{-2em}
  \title{Text Mining Techniques for Knowledge Extraction from Technical Documents}
  \pretitle{\vspace{\droptitle}\centering\huge}
  \posttitle{\par}
  \author{Filippo Chiarello}
  \preauthor{\centering\large\emph}
  \postauthor{\par}
  \predate{\centering\large\emph}
  \postdate{\par}
  \date{2018-08-29}

\usepackage{booktabs}

\begin{document}
\maketitle

{
\setcounter{tocdepth}{1}
\tableofcontents
}
\chapter{Introduction}\label{intro}

\section{Goal}\label{goal}

\section{Problem Solutions}\label{problem-solutions}

\section{Solutions}\label{solutions}

\section{Challenges: Understanding and
programming}\label{challenges-understanding-and-programming}

\subsection{Understand}\label{understand}

\subsection{Program}\label{program}

\section{Research Questions}\label{research-questions}

\chapter{State of the Art}\label{state-of-the-art}

Here is a review of existing methods.

\section{Tools}\label{tools}

\subsection{Program}\label{program-1}

\subsection{Understand}\label{understand-1}

\subsection{Import}\label{import}

\subsubsection{Retrieval}\label{retrieval}

\subsection{Tidy}\label{tidy}

\subsection{Transform}\label{transform}

\subsubsection{Preprocessing}\label{preprocessing}

\subsection{Model}\label{model}

\subsubsection{Pattern Analysis}\label{pattern-analysis}

\subsubsection{Sentiment Analysis}\label{sentiment-analysis}

\subsubsection{Named Entity Recognition}\label{named-entity-recognition}

\subsubsection{Network Analysis}\label{network-analysis}

\subsubsection{Topic Modelling}\label{topic-modelling}

\subsection{Visualize}\label{visualize}

\subsection{Comunicate}\label{comunicate}

\section{Documents}\label{documents}

\subsection{Patents}\label{patents}

\subsection{Papers}\label{papers}

\subsection{Projects}\label{projects}

\subsection{Wikipedia}\label{wikipedia}

\subsection{Twitter}\label{twitter}

\subsection{Job Profiles}\label{job-profiles}

\chapter{Methods}\label{methods}

We describe our methods in this chapter.

\section{Patents}\label{patents-1}

\section{Papers}\label{papers-1}

\section{Projects}\label{projects-1}

\section{Wikipedia}\label{wikipedia-1}

\section{Twitter}\label{twitter-1}

\section{Job Profiles}\label{job-profiles-1}

\chapter{Applications and Results}\label{applications-and-results}

Some \emph{significant} applications are demonstrated in this chapter.

\section{Patents}\label{patents-2}

\section{Papers}\label{papers-2}

\section{Projects}\label{projects-2}

\section{Wikipedia}\label{wikipedia-2}

\section{Twitter}\label{twitter-2}

\section{Job Profiles}\label{job-profiles-2}

\chapter{Future Developments}\label{future-developments}

Decide the way in wich devide the chapter. Tools or functions?

\section{Research and Development}\label{research-and-development}

\section{Marketing}\label{marketing}

\section{Human Resources}\label{human-resources}

\chapter{Final Words}\label{final-words}

We have finished a nice book.

\bibliography{bib\_book.bib,bib\_papers.bib,bib\_web.bib,bib\_packages.bib}


\end{document}
